% Options for packages loaded elsewhere
\PassOptionsToPackage{unicode}{hyperref}
\PassOptionsToPackage{hyphens}{url}
%
\documentclass[
  ignorenonframetext,
]{beamer}
\usepackage{pgfpages}
\setbeamertemplate{caption}[numbered]
\setbeamertemplate{caption label separator}{: }
\setbeamercolor{caption name}{fg=normal text.fg}
\beamertemplatenavigationsymbolsempty
% Prevent slide breaks in the middle of a paragraph
\widowpenalties 1 10000
\raggedbottom
\setbeamertemplate{part page}{
  \centering
  \begin{beamercolorbox}[sep=16pt,center]{part title}
    \usebeamerfont{part title}\insertpart\par
  \end{beamercolorbox}
}
\setbeamertemplate{section page}{
  \centering
  \begin{beamercolorbox}[sep=12pt,center]{part title}
    \usebeamerfont{section title}\insertsection\par
  \end{beamercolorbox}
}
\setbeamertemplate{subsection page}{
  \centering
  \begin{beamercolorbox}[sep=8pt,center]{part title}
    \usebeamerfont{subsection title}\insertsubsection\par
  \end{beamercolorbox}
}
\AtBeginPart{
  \frame{\partpage}
}
\AtBeginSection{
  \ifbibliography
  \else
    \frame{\sectionpage}
  \fi
}
\AtBeginSubsection{
  \frame{\subsectionpage}
}
\usepackage{amsmath,amssymb}
\usepackage{iftex}
\ifPDFTeX
  \usepackage[T1]{fontenc}
  \usepackage[utf8]{inputenc}
  \usepackage{textcomp} % provide euro and other symbols
\else % if luatex or xetex
  \usepackage{unicode-math} % this also loads fontspec
  \defaultfontfeatures{Scale=MatchLowercase}
  \defaultfontfeatures[\rmfamily]{Ligatures=TeX,Scale=1}
\fi
\usepackage{lmodern}
\usetheme[]{CambridgeUS}
\usecolortheme{dolphin}
\usefonttheme{structurebold}
\ifPDFTeX\else
  % xetex/luatex font selection
\fi
% Use upquote if available, for straight quotes in verbatim environments
\IfFileExists{upquote.sty}{\usepackage{upquote}}{}
\IfFileExists{microtype.sty}{% use microtype if available
  \usepackage[]{microtype}
  \UseMicrotypeSet[protrusion]{basicmath} % disable protrusion for tt fonts
}{}
\makeatletter
\@ifundefined{KOMAClassName}{% if non-KOMA class
  \IfFileExists{parskip.sty}{%
    \usepackage{parskip}
  }{% else
    \setlength{\parindent}{0pt}
    \setlength{\parskip}{6pt plus 2pt minus 1pt}}
}{% if KOMA class
  \KOMAoptions{parskip=half}}
\makeatother
\usepackage{xcolor}
\newif\ifbibliography
\usepackage{color}
\usepackage{fancyvrb}
\newcommand{\VerbBar}{|}
\newcommand{\VERB}{\Verb[commandchars=\\\{\}]}
\DefineVerbatimEnvironment{Highlighting}{Verbatim}{commandchars=\\\{\}}
% Add ',fontsize=\small' for more characters per line
\usepackage{framed}
\definecolor{shadecolor}{RGB}{248,248,248}
\newenvironment{Shaded}{\begin{snugshade}}{\end{snugshade}}
\newcommand{\AlertTok}[1]{\textcolor[rgb]{0.94,0.16,0.16}{#1}}
\newcommand{\AnnotationTok}[1]{\textcolor[rgb]{0.56,0.35,0.01}{\textbf{\textit{#1}}}}
\newcommand{\AttributeTok}[1]{\textcolor[rgb]{0.13,0.29,0.53}{#1}}
\newcommand{\BaseNTok}[1]{\textcolor[rgb]{0.00,0.00,0.81}{#1}}
\newcommand{\BuiltInTok}[1]{#1}
\newcommand{\CharTok}[1]{\textcolor[rgb]{0.31,0.60,0.02}{#1}}
\newcommand{\CommentTok}[1]{\textcolor[rgb]{0.56,0.35,0.01}{\textit{#1}}}
\newcommand{\CommentVarTok}[1]{\textcolor[rgb]{0.56,0.35,0.01}{\textbf{\textit{#1}}}}
\newcommand{\ConstantTok}[1]{\textcolor[rgb]{0.56,0.35,0.01}{#1}}
\newcommand{\ControlFlowTok}[1]{\textcolor[rgb]{0.13,0.29,0.53}{\textbf{#1}}}
\newcommand{\DataTypeTok}[1]{\textcolor[rgb]{0.13,0.29,0.53}{#1}}
\newcommand{\DecValTok}[1]{\textcolor[rgb]{0.00,0.00,0.81}{#1}}
\newcommand{\DocumentationTok}[1]{\textcolor[rgb]{0.56,0.35,0.01}{\textbf{\textit{#1}}}}
\newcommand{\ErrorTok}[1]{\textcolor[rgb]{0.64,0.00,0.00}{\textbf{#1}}}
\newcommand{\ExtensionTok}[1]{#1}
\newcommand{\FloatTok}[1]{\textcolor[rgb]{0.00,0.00,0.81}{#1}}
\newcommand{\FunctionTok}[1]{\textcolor[rgb]{0.13,0.29,0.53}{\textbf{#1}}}
\newcommand{\ImportTok}[1]{#1}
\newcommand{\InformationTok}[1]{\textcolor[rgb]{0.56,0.35,0.01}{\textbf{\textit{#1}}}}
\newcommand{\KeywordTok}[1]{\textcolor[rgb]{0.13,0.29,0.53}{\textbf{#1}}}
\newcommand{\NormalTok}[1]{#1}
\newcommand{\OperatorTok}[1]{\textcolor[rgb]{0.81,0.36,0.00}{\textbf{#1}}}
\newcommand{\OtherTok}[1]{\textcolor[rgb]{0.56,0.35,0.01}{#1}}
\newcommand{\PreprocessorTok}[1]{\textcolor[rgb]{0.56,0.35,0.01}{\textit{#1}}}
\newcommand{\RegionMarkerTok}[1]{#1}
\newcommand{\SpecialCharTok}[1]{\textcolor[rgb]{0.81,0.36,0.00}{\textbf{#1}}}
\newcommand{\SpecialStringTok}[1]{\textcolor[rgb]{0.31,0.60,0.02}{#1}}
\newcommand{\StringTok}[1]{\textcolor[rgb]{0.31,0.60,0.02}{#1}}
\newcommand{\VariableTok}[1]{\textcolor[rgb]{0.00,0.00,0.00}{#1}}
\newcommand{\VerbatimStringTok}[1]{\textcolor[rgb]{0.31,0.60,0.02}{#1}}
\newcommand{\WarningTok}[1]{\textcolor[rgb]{0.56,0.35,0.01}{\textbf{\textit{#1}}}}
\setlength{\emergencystretch}{3em} % prevent overfull lines
\providecommand{\tightlist}{%
  \setlength{\itemsep}{0pt}\setlength{\parskip}{0pt}}
\setcounter{secnumdepth}{-\maxdimen} % remove section numbering
\ifLuaTeX
  \usepackage{selnolig}  % disable illegal ligatures
\fi
\IfFileExists{bookmark.sty}{\usepackage{bookmark}}{\usepackage{hyperref}}
\IfFileExists{xurl.sty}{\usepackage{xurl}}{} % add URL line breaks if available
\urlstyle{same}
\hypersetup{
  pdftitle={Lois de probabilités discrètes sur R},
  pdfauthor={Julien Parfait Bidias Assala},
  hidelinks,
  pdfcreator={LaTeX via pandoc}}

\title{Lois de probabilités discrètes sur R}
\author{Julien Parfait Bidias Assala}
\date{2024-08-21}

\begin{document}
\frame{\titlepage}

\begin{frame}{Background}
\protect\hypertarget{background}{}
\begin{itemize}
\item
  Les lois de probabilités discrètes en statistiques permettent de
  modéliser des situations où les variables aléatoires prennent un
  nombre fini ou dénombrable de valeurs distinctes. Ces lois sont
  couramment utilisées pour décrire des phénomènes où les résultats
  possibles sont limités à des catégories distinctes ou des événements
  discrets.
\item
  Par exemple, la loi binomiale permet de modéliser le nombre de succès
  dans une série d'essais indépendants, tandis que la loi de Poisson est
  utilisée pour décrire le nombre d'événements survenant dans un
  intervalle de temps ou d'espace fixe. D'autres exemples incluent la
  loi géométrique, qui mesure le nombre d'essais avant le premier
  succès, et la loi hypergéométrique, utilisée pour des échantillons
  sans remise.
\end{itemize}
\end{frame}

\begin{frame}{Lois de probabilités discrètes}
\protect\hypertarget{lois-de-probabilituxe9s-discruxe8tes}{}
Comment modéliser simplement, sans packages et de façon tout à fait
compréhensible toutes ces différentes lois ? Tel est l'objet de la
présentation.

Nous présentons la loi :

\begin{itemize}
\tightlist
\item
  De Bernoulli
\item
  Binomiale
\item
  De Poisson
\item
  Géométrique
\item
  Uniforme discrète
\item
  Hypergéométrique
\item
  Binomiale négative
\end{itemize}
\end{frame}

\begin{frame}[fragile]{Loi de Bernoulli}
\protect\hypertarget{loi-de-bernouilli}{}

X suit une loi de Bernoulli lorsque :

\[P(X=k) = p^k (1-p)^{1-k} \quad avec \quad  k=0,1 \quad et \quad p \in ]0,1[\]

D'où le programme ci-dessous.

\begin{Shaded}
\begin{Highlighting}[]
\NormalTok{bernoulli }\OtherTok{=} \ControlFlowTok{function}\NormalTok{(p,k)\{}
  \ControlFlowTok{if}\NormalTok{ ( (k}\SpecialCharTok{==}\DecValTok{0} \SpecialCharTok{|}\NormalTok{ k}\SpecialCharTok{==}\DecValTok{1}\NormalTok{) }\SpecialCharTok{\&}\NormalTok{ (p}\SpecialCharTok{\textgreater{}}\DecValTok{0} \SpecialCharTok{\&}\NormalTok{ p}\SpecialCharTok{\textless{}}\DecValTok{1}\NormalTok{) )\{}
\NormalTok{        probabilite }\OtherTok{=}\NormalTok{ p}\SpecialCharTok{**}\NormalTok{k }\SpecialCharTok{*}\NormalTok{ (}\DecValTok{1}\SpecialCharTok{{-}}\NormalTok{p)}\SpecialCharTok{**}\NormalTok{(}\DecValTok{1}\SpecialCharTok{{-}}\NormalTok{k)}
        \FunctionTok{print}\NormalTok{(probabilite)}
\NormalTok{  \} }\ControlFlowTok{else}
    \FunctionTok{print}\NormalTok{(}\StringTok{"Entrer des valeurs correctes de k ou de p"}\NormalTok{)}
\NormalTok{\}}
\end{Highlighting}
\end{Shaded}
\end{frame}

\begin{frame}[fragile]{Loi de Bernoulli}
\protect\hypertarget{loi-de-bernouilli-1}{}
Supposons que X suit une loi de Bernoulli
avec \(p=0.4\). On veut calculer \(P(X=0)\) et \(P(X=1)\).

\begin{Shaded}
\begin{Highlighting}[]
\FunctionTok{bernoulli}\NormalTok{(}\FloatTok{0.4}\NormalTok{, }\DecValTok{0}\NormalTok{)}
\end{Highlighting}
\end{Shaded}

\begin{verbatim}
[1] 0.6
\end{verbatim}

\begin{Shaded}
\begin{Highlighting}[]
\FunctionTok{bernoulli}\NormalTok{(}\FloatTok{0.4}\NormalTok{, }\DecValTok{1}\NormalTok{)}
\end{Highlighting}
\end{Shaded}

\begin{verbatim}
[1] 0.4
\end{verbatim}

\begin{Shaded}
\begin{Highlighting}[]
\FunctionTok{bernoulli}\NormalTok{(}\FloatTok{0.4}\NormalTok{, }\DecValTok{3}\NormalTok{)}
\end{Highlighting}
\end{Shaded}

\begin{verbatim}
[1] "Entrer des valeurs correctes de k ou de p"
\end{verbatim}
\end{frame}

\begin{frame}{Espérance, variance et Fonction de répartition}
\protect\hypertarget{espuxe9rance-variance-et-fonction-de-ruxe9partition}{}
Pour la moyenne et la variance il suffit de prendre :

\[E(X) = p\] \[Var(X) = pq\] La construction de la fonction de
répartition passe par la relation :

\[F_X (k) = P(X\le k)\] On peut passer aussi par la relation :

\[F_X (k) = P(X\le k) = \sum_{i=0}^k p^k (1-p)^{1-k}\]

Ainsi, si \(k=0\) on a \(1-p\). Par contre, si \(k=1\) on aura 1.
\end{frame}

\begin{frame}[fragile]{Espérance, variance et Fonction de répartition}
\protect\hypertarget{espuxe9rance-variance-et-fonction-de-ruxe9partition-1}{}
\begin{Shaded}
\begin{Highlighting}[]
\NormalTok{repartition.bern }\OtherTok{=} \ControlFlowTok{function}\NormalTok{(p,k)\{}
\NormalTok{  i     }\OtherTok{=} \DecValTok{0}
\NormalTok{  somme }\OtherTok{=} \DecValTok{0}
  \ControlFlowTok{while}\NormalTok{(i}\SpecialCharTok{\textless{}=}\NormalTok{k)\{}
    \ControlFlowTok{if}\NormalTok{ ( (k}\SpecialCharTok{==}\DecValTok{0} \SpecialCharTok{|}\NormalTok{ k}\SpecialCharTok{==}\DecValTok{1}\NormalTok{) }\SpecialCharTok{\&}\NormalTok{ (p}\SpecialCharTok{\textgreater{}}\DecValTok{0} \SpecialCharTok{\&}\NormalTok{ p}\SpecialCharTok{\textless{}}\DecValTok{1}\NormalTok{) )\{}
\NormalTok{      probabilite }\OtherTok{=}\NormalTok{ p}\SpecialCharTok{**}\NormalTok{i }\SpecialCharTok{*}\NormalTok{ (}\DecValTok{1}\SpecialCharTok{{-}}\NormalTok{p)}\SpecialCharTok{**}\NormalTok{(}\DecValTok{1}\SpecialCharTok{{-}}\NormalTok{i)}
\NormalTok{      somme }\OtherTok{=}\NormalTok{ somme }\SpecialCharTok{+}\NormalTok{ probabilite}
\NormalTok{      i }\OtherTok{=}\NormalTok{ i}\SpecialCharTok{+}\DecValTok{1}
\NormalTok{      \} }\ControlFlowTok{else}
        \FunctionTok{print}\NormalTok{(}\StringTok{"Entrer des valeurs correctes de k ou de p"}\NormalTok{)}
\NormalTok{  \}}
  \FunctionTok{print}\NormalTok{(somme)}
\NormalTok{\}}
\end{Highlighting}
\end{Shaded}
\end{frame}

\begin{frame}[fragile]{Fonction de répartition}
\protect\hypertarget{fonction-de-ruxe9partition}{}

Application : 


Calculons \(P(X \le 0)\) et \(P(X \le 1)\) pour une loi de Bernoulli de
paramètre \(p=0.4\).

\begin{Shaded}
\begin{Highlighting}[]
\FunctionTok{repartition.bern}\NormalTok{(}\FloatTok{0.4}\NormalTok{, }\DecValTok{0}\NormalTok{) }
\end{Highlighting}
\end{Shaded}

\begin{verbatim}
[1] 0.6
\end{verbatim}

\begin{Shaded}
\begin{Highlighting}[]
\FunctionTok{repartition.bern}\NormalTok{(}\FloatTok{0.4}\NormalTok{, }\DecValTok{1}\NormalTok{)}
\end{Highlighting}
\end{Shaded}

\begin{verbatim}
[1] 1
\end{verbatim}

La fonction de répartition associée à X prend ainsi deux valeurs quelque
soit la valeur du paramètre p.
\end{frame}

\begin{frame}{Fonction de répartition}
\protect\hypertarget{fonction-de-ruxe9partition-1}{}
\begin{center}\includegraphics{lois_discretes_files/figure-beamer/unnamed-chunk-5-1} \end{center}
\end{frame}

\begin{frame}[fragile]{Loi Binomiale}
\protect\hypertarget{loi-binomiale}{}
\(P(X=k)\) pour une loi binomiale :
\(\quad avec \quad k=0,1, \cdots, n \quad ; \quad n \in N^* \quad p \in ]0,1[ \quad ; \quad k\le n\)

\[P(X=k) = C_n^k p^k (1-p)^{n-k}\]

\begin{Shaded}
\begin{Highlighting}[]
\NormalTok{binome }\OtherTok{=} \ControlFlowTok{function}\NormalTok{(n, p, k)\{}
  \ControlFlowTok{if}\NormalTok{(k}\SpecialCharTok{\textless{}=}\NormalTok{n }\SpecialCharTok{\&}\NormalTok{ (p}\SpecialCharTok{\textgreater{}}\DecValTok{0} \SpecialCharTok{\&\&}\NormalTok{ p}\SpecialCharTok{\textless{}}\DecValTok{1}\NormalTok{) }\SpecialCharTok{\&}\NormalTok{ n}\SpecialCharTok{\textgreater{}}\DecValTok{0} \SpecialCharTok{\&\&} \FunctionTok{is.numeric}\NormalTok{(n) }\SpecialCharTok{\&\&}\NormalTok{ n}\SpecialCharTok{\%\%}\DecValTok{1}\SpecialCharTok{==}\DecValTok{0}\NormalTok{)\{}
\NormalTok{    combinaison }\OtherTok{=} \FunctionTok{factorial}\NormalTok{(n)}\SpecialCharTok{/}\NormalTok{(}\FunctionTok{factorial}\NormalTok{(k)}\SpecialCharTok{*}\FunctionTok{factorial}\NormalTok{(n}\SpecialCharTok{{-}}\NormalTok{k))}
\NormalTok{    probabilite  }\OtherTok{=}\NormalTok{ combinaison }\SpecialCharTok{*}\NormalTok{ p}\SpecialCharTok{**}\NormalTok{k }\SpecialCharTok{*}\NormalTok{ (}\DecValTok{1}\SpecialCharTok{{-}}\NormalTok{p)}\SpecialCharTok{**}\NormalTok{(n}\SpecialCharTok{{-}}\NormalTok{k)}
    \FunctionTok{print}\NormalTok{(probabilite)}
\NormalTok{  \} }\ControlFlowTok{else}\NormalTok{\{}
    \FunctionTok{print}\NormalTok{(}\StringTok{"Entrer des valeurs correctes de n, p et k"}\NormalTok{)}
\NormalTok{  \}}
\NormalTok{\}}
\end{Highlighting}
\end{Shaded}
\end{frame}

\begin{frame}[fragile]{Loi Binomiale}
\protect\hypertarget{loi-binomiale-1}{}

Application : \(P(X=1)\) pour une loi \(B(n=2, p=0.5)\) :

\begin{Shaded}
\begin{Highlighting}[]
\NormalTok{k }\OtherTok{=} \DecValTok{1}
\NormalTok{n }\OtherTok{=} \DecValTok{2}
\NormalTok{p }\OtherTok{=} \FloatTok{0.5}
\FunctionTok{binome}\NormalTok{(n, p, k) }
\end{Highlighting}
\end{Shaded}

\begin{verbatim}
[1] 0.5
\end{verbatim}

Si on veut calculer par exmple \(P(X=4)\) avec X qui suit une loi
\(B(n=2, p=0.5)\) alors on aura :

\begin{Shaded}
\begin{Highlighting}[]
\FunctionTok{binome}\NormalTok{(}\DecValTok{2}\NormalTok{, }\FloatTok{0.5}\NormalTok{, }\DecValTok{4}\NormalTok{)}
\end{Highlighting}
\end{Shaded}

\begin{verbatim}
[1] "Entrer des valeurs correctes de n, p et k"
\end{verbatim}

Ce qui est tout a fait normal car \(k\) ne peut dépasser \(n\).
\end{frame}

\begin{frame}[fragile]{Espérance, variance et fonction de répartition}
\protect\hypertarget{espuxe9rance-variance-et-fonction-de-ruxe9partition-2}{}
\(E(X)=np\) et \(Var(X)=np(1-p)\) on peut s'amuser à programmer une
petite fonction qui renvoit ces caractéristiques :

\begin{Shaded}
\begin{Highlighting}[]
\NormalTok{carac.binom }\OtherTok{=} \ControlFlowTok{function}\NormalTok{(n, p)\{}
  \ControlFlowTok{if}\NormalTok{(n}\SpecialCharTok{\textgreater{}}\DecValTok{0} \SpecialCharTok{\&\&}\NormalTok{ (p}\SpecialCharTok{\textgreater{}}\DecValTok{0} \SpecialCharTok{\&}\NormalTok{ p}\SpecialCharTok{\textless{}}\DecValTok{1}\NormalTok{) }\SpecialCharTok{\&\&} \FunctionTok{is.numeric}\NormalTok{(n) }\SpecialCharTok{\&\&}\NormalTok{ n}\SpecialCharTok{\%\%}\DecValTok{1}\SpecialCharTok{==}\DecValTok{0}\NormalTok{)\{}
\NormalTok{      esperance }\OtherTok{=}\NormalTok{ n}\SpecialCharTok{*}\NormalTok{p}
\NormalTok{      variance  }\OtherTok{=}\NormalTok{ n}\SpecialCharTok{*}\NormalTok{p}\SpecialCharTok{*}\NormalTok{(}\DecValTok{1}\SpecialCharTok{{-}}\NormalTok{p)}
      \FunctionTok{return}\NormalTok{(}\FunctionTok{list}\NormalTok{(}\AttributeTok{Esperance =}\NormalTok{ esperance, }\AttributeTok{Variance=}\NormalTok{variance))}
\NormalTok{  \} }\ControlFlowTok{else}\NormalTok{\{}
    \FunctionTok{print}\NormalTok{(}\StringTok{"Entrer des valeurs correctes de n ou de p"}\NormalTok{)}
\NormalTok{  \}}
\NormalTok{\}}
\end{Highlighting}
\end{Shaded}
\end{frame}

\begin{frame}[fragile]{Exemple}
\protect\hypertarget{exemple}{}
si \(X\) suit une B(4, 0.8) alors :

\begin{Shaded}
\begin{Highlighting}[]
\FunctionTok{carac.binom}\NormalTok{(}\DecValTok{4}\NormalTok{, }\FloatTok{0.8}\NormalTok{)}
\end{Highlighting}
\end{Shaded}

\begin{verbatim}
Esperance
[1] 3.2
 
Variance
[1] 0.64
\end{verbatim}

Pour la fonction de répartition, on peut passer la relation :

\[F_X (k) = P(X\le k) = \sum_{i=0}^k P(X=i) = \sum_{i=0}^k C_n^i p^i (1-p)^{1-i}\]
\end{frame}

\begin{frame}[fragile]{Fonction de répartition de la loi Binomiale}
\protect\hypertarget{fonction-de-ruxe9partition-de-la-loi-binomiale}{}
\begin{Shaded}
\begin{Highlighting}[]
\NormalTok{repartition.binom }\OtherTok{=} \ControlFlowTok{function}\NormalTok{(n, p, k)\{}
\NormalTok{  i }\OtherTok{=} \DecValTok{0}
\NormalTok{  somme }\OtherTok{=} \DecValTok{0}
  \ControlFlowTok{if}\NormalTok{(k}\SpecialCharTok{\textless{}=}\NormalTok{n }\SpecialCharTok{\&}\NormalTok{ (p}\SpecialCharTok{\textgreater{}}\DecValTok{0} \SpecialCharTok{\&}\NormalTok{ p}\SpecialCharTok{\textless{}}\DecValTok{1}\NormalTok{) }\SpecialCharTok{\&}\NormalTok{ n}\SpecialCharTok{\textgreater{}}\DecValTok{0} \SpecialCharTok{\&} \FunctionTok{is.numeric}\NormalTok{(n) }\SpecialCharTok{\&}\NormalTok{ n}\SpecialCharTok{\%\%}\DecValTok{1}\SpecialCharTok{==}\DecValTok{0}\NormalTok{)\{}
    \ControlFlowTok{while}\NormalTok{(i}\SpecialCharTok{\textless{}=}\NormalTok{k)\{}
\NormalTok{      combinaison}\OtherTok{=}\FunctionTok{factorial}\NormalTok{(n)}\SpecialCharTok{/}\NormalTok{(}\FunctionTok{factorial}\NormalTok{(i)}\SpecialCharTok{*}\FunctionTok{factorial}\NormalTok{(n}\SpecialCharTok{{-}}\NormalTok{i))}
\NormalTok{      probabilite}\OtherTok{=}\NormalTok{ combinaison }\SpecialCharTok{*}\NormalTok{ p}\SpecialCharTok{**}\NormalTok{i }\SpecialCharTok{*}\NormalTok{ (}\DecValTok{1}\SpecialCharTok{{-}}\NormalTok{p)}\SpecialCharTok{**}\NormalTok{(n}\SpecialCharTok{{-}}\NormalTok{i)}
\NormalTok{      somme }\OtherTok{=}\NormalTok{ somme }\SpecialCharTok{+}\NormalTok{ probabilite}
\NormalTok{      i }\OtherTok{=}\NormalTok{ i }\SpecialCharTok{+} \DecValTok{1}
\NormalTok{      \}}
\NormalTok{    \}}\ControlFlowTok{else}\NormalTok{\{}
    \FunctionTok{print}\NormalTok{(}\StringTok{"Entrer des valeurs correctes de n, de p ou de k"}\NormalTok{)}
\NormalTok{    \}}
 \FunctionTok{print}\NormalTok{(somme)}
\NormalTok{\}}
\end{Highlighting}
\end{Shaded}
\end{frame}

\begin{frame}[fragile]{Exemple}
\protect\hypertarget{exemple-1}{}

Application : 


Calculons \(P(X \le 1)\) pour une loi \(B(2, 0.5)\). On sait que
\(P(X \le 1) = P(X=0) + P(X=1)\). Par conséquent,
\(P(X \le 1) = 0.25 + 0.5 = 0.75\).

\begin{Shaded}
\begin{Highlighting}[]
\NormalTok{n }\OtherTok{=} \DecValTok{2}
\NormalTok{p }\OtherTok{=} \FloatTok{0.5}
\NormalTok{k }\OtherTok{=} \DecValTok{1}
\FunctionTok{repartition.binom}\NormalTok{ (n, p, k)}
\end{Highlighting}
\end{Shaded}

\begin{verbatim}
[1] 0.75
\end{verbatim}
\end{frame}

\begin{frame}{Fonction de répartition d'une loi B(100, 0.5) :}
\protect\hypertarget{fonction-de-ruxe9partition-dune-loi-b100-0.5}{}
\begin{center}\includegraphics{lois_discretes_files/figure-beamer/unnamed-chunk-14-1} \end{center}
\end{frame}

\begin{frame}[fragile]{Lois de Poisson}
\protect\hypertarget{lois-de-poisson}{}
La P(X=k) lorsque X suit une loi de Poisson est donnée par :

\[P(X=k) = \frac{e^{-\lambda} \lambda^{k}}{k!}\] Avec
\(k=0, 1,\cdots,\infty\) et \(\lambda>0\). \(E(X)=\lambda\) et
\(Var(X)=\lambda\).

\begin{Shaded}
\begin{Highlighting}[]
\NormalTok{poisson }\OtherTok{=} \ControlFlowTok{function}\NormalTok{(lambda, k)\{}
  \ControlFlowTok{if}\NormalTok{(lambda}\SpecialCharTok{\textgreater{}}\DecValTok{0} \SpecialCharTok{\&\&}\NormalTok{ k}\SpecialCharTok{\textgreater{}}\DecValTok{0} \SpecialCharTok{\&\&} \FunctionTok{is.numeric}\NormalTok{(k) }\SpecialCharTok{\&\&}\NormalTok{ k}\SpecialCharTok{\%\%}\DecValTok{1}\SpecialCharTok{==}\DecValTok{0}\NormalTok{)\{}
\NormalTok{    probabilite }\OtherTok{=}\NormalTok{ (}\FunctionTok{exp}\NormalTok{(}\SpecialCharTok{{-}}\NormalTok{lambda)}\SpecialCharTok{*}\NormalTok{lambda}\SpecialCharTok{**}\NormalTok{k)}\SpecialCharTok{/}\FunctionTok{factorial}\NormalTok{(k)}
    \FunctionTok{print}\NormalTok{(probabilite)}
\NormalTok{  \} }\ControlFlowTok{else}\NormalTok{ \{}
    \FunctionTok{print}\NormalTok{(}\StringTok{"Entrer des valeurs correctes de k ou de lambda"}\NormalTok{)}
\NormalTok{  \}}
\NormalTok{\}}
\end{Highlighting}
\end{Shaded}
\end{frame}

\begin{frame}[fragile]{Exemple d'application}
\protect\hypertarget{exemple-2}{}
\begin{Shaded}
\begin{Highlighting}[]
\FunctionTok{poisson}\NormalTok{(}\DecValTok{10}\NormalTok{, }\DecValTok{2}\NormalTok{)}
\end{Highlighting}
\end{Shaded}

\begin{verbatim}
[1] 0.002269996
\end{verbatim}

Pour \(n\) assez grand et \(p\) très petit, on peut approximer une loi
binomiale par une loi de Poisson. Supposons que l'on veuille calculer :
P(X=2) avec X suivant une loi \(B(100, 0.01)\). On pose donc
\(np=\lambda\) et on calcule aussi P(X=2) avec X suivant une loi
\(P(\lambda = np)\)

\begin{Shaded}
\begin{Highlighting}[]
\NormalTok{b1 }\OtherTok{=} \FunctionTok{binome}\NormalTok{(}\DecValTok{100}\NormalTok{, }\FloatTok{0.01}\NormalTok{, }\DecValTok{2}\NormalTok{)}
\end{Highlighting}
\end{Shaded}

\begin{verbatim}
[1] 0.1848648
\end{verbatim}

\begin{Shaded}
\begin{Highlighting}[]
\NormalTok{p1 }\OtherTok{=} \FunctionTok{poisson}\NormalTok{(}\DecValTok{100}\SpecialCharTok{*}\FloatTok{0.01}\NormalTok{, }\DecValTok{2}\NormalTok{)}
\end{Highlighting}
\end{Shaded}

\begin{verbatim}
[1] 0.1839397
\end{verbatim}
\end{frame}

\begin{frame}[fragile]{Fonction de répartition de la loi de Poisson}
\protect\hypertarget{fonction-de-ruxe9partition-de-la-loi-de-poisson}{}
On applique la formule suivante :
\(F_X (k) = P(X\le k) = \sum_{i=0}^k P(X=i) = \sum_{i=0}^k \frac{e^{-\lambda} \lambda^{i}}{i!}\)

\begin{Shaded}
\begin{Highlighting}[]
\NormalTok{repartition.poiss }\OtherTok{=} \ControlFlowTok{function}\NormalTok{(lambda, k)\{}
\NormalTok{  i }\OtherTok{=} \DecValTok{0}
\NormalTok{  somme }\OtherTok{=} \DecValTok{0}
  \ControlFlowTok{if}\NormalTok{(lambda}\SpecialCharTok{\textgreater{}}\DecValTok{0} \SpecialCharTok{\&\&}\NormalTok{ k}\SpecialCharTok{\textgreater{}}\DecValTok{0} \SpecialCharTok{\&\&} \FunctionTok{is.numeric}\NormalTok{(k) }\SpecialCharTok{\&\&}\NormalTok{ k}\SpecialCharTok{\%\%}\DecValTok{1}\SpecialCharTok{==}\DecValTok{0}\NormalTok{)\{}
    \ControlFlowTok{while}\NormalTok{(i}\SpecialCharTok{\textless{}=}\NormalTok{k)\{}
\NormalTok{      probabilite }\OtherTok{=}\NormalTok{ (}\FunctionTok{exp}\NormalTok{(}\SpecialCharTok{{-}}\NormalTok{lambda)}\SpecialCharTok{*}\NormalTok{lambda}\SpecialCharTok{**}\NormalTok{i)}\SpecialCharTok{/}\FunctionTok{factorial}\NormalTok{(i)}
\NormalTok{      somme }\OtherTok{=}\NormalTok{ somme }\SpecialCharTok{+}\NormalTok{ probabilite}
\NormalTok{      i }\OtherTok{=}\NormalTok{ i }\SpecialCharTok{+} \DecValTok{1}
\NormalTok{      \}}
\NormalTok{    \}}\ControlFlowTok{else}\NormalTok{\{}
    \FunctionTok{print}\NormalTok{(}\StringTok{"Entrer des valeurs correctes de n, de p ou de k"}\NormalTok{)}
\NormalTok{    \}}
 \FunctionTok{print}\NormalTok{(somme)}
\NormalTok{\}}
\end{Highlighting}
\end{Shaded}
\end{frame}

\begin{frame}{Fonction de répartition de la loi de poisson}
\protect\hypertarget{fonction-de-ruxe9partition-de-la-loi-de-poisson-1}{}
\begin{center}\includegraphics{lois_discretes_files/figure-beamer/unnamed-chunk-20-1} \end{center}
\end{frame}

\begin{frame}[fragile]{Loi Géométrique}
\protect\hypertarget{loi-guxe9omuxe9trique}{}
X suit loi géométrique si et seulement si sa loi de probabilité est donnée par :

\[P(X=k) = p(1-p)^{k-1}\] Avec \(k=1, 2, \cdots, \infty\) et
\(p \in ]0 , 1[\)

\begin{Shaded}
\begin{Highlighting}[]
\NormalTok{geometrique }\OtherTok{=} \ControlFlowTok{function}\NormalTok{(p, k)\{}
  \ControlFlowTok{if}\NormalTok{(k}\SpecialCharTok{\textgreater{}=}\DecValTok{0} \SpecialCharTok{\&}\NormalTok{  (p}\SpecialCharTok{\textgreater{}}\DecValTok{0} \SpecialCharTok{\&}\NormalTok{ p}\SpecialCharTok{\textless{}}\DecValTok{1}\NormalTok{) }\SpecialCharTok{\&}  \FunctionTok{is.numeric}\NormalTok{(k) }\SpecialCharTok{\&}\NormalTok{ k}\SpecialCharTok{\%\%}\DecValTok{1}\SpecialCharTok{==}\DecValTok{0}\NormalTok{)\{}
\NormalTok{    probabilite }\OtherTok{=}\NormalTok{ p}\SpecialCharTok{*}\NormalTok{(}\DecValTok{1}\SpecialCharTok{{-}}\NormalTok{p)}\SpecialCharTok{**}\NormalTok{(k}\DecValTok{{-}1}\NormalTok{)}
    \FunctionTok{print}\NormalTok{(probabilite)}
\NormalTok{  \}}\ControlFlowTok{else}\NormalTok{\{}
    \FunctionTok{print}\NormalTok{(}\StringTok{"Entrer des valeurs correctes de p ou de k"}\NormalTok{)}
\NormalTok{  \}}
\NormalTok{\}}
\end{Highlighting}
\end{Shaded}
\end{frame}

\begin{frame}[fragile]{Exemple :}
\protect\hypertarget{exemple-3}{}
Calculons \(P(X=3)\) avec \(X\) suivant une loi \(G(p=0.7)\).

\begin{Shaded}
\begin{Highlighting}[]
\FunctionTok{geometrique}\NormalTok{(}\FloatTok{0.7}\NormalTok{, }\DecValTok{3}\NormalTok{)}
\end{Highlighting}
\end{Shaded}

\begin{verbatim}
[1] 0.063
\end{verbatim}

Les caractéristiques d'une loi géométrique sont : \(E(X)=\frac{1}{p}\)
et \(Var(X)=\frac{1-p}{p^2}\).
\end{frame}

\begin{frame}[fragile]{Fonction de répartition de la loi géométrique}
\protect\hypertarget{fonction-de-ruxe9partition-de-la-loi-guxe9omuxe9trique}{}
\(F_X (k) = P(X\le k) = \sum_{i=0}^k P(X=i) = \sum_{i=0}^k p(1-p)^{i-1}\)

\begin{Shaded}
\begin{Highlighting}[]
\NormalTok{repartition.geom }\OtherTok{=} \ControlFlowTok{function}\NormalTok{(p, k)\{}
\NormalTok{  i }\OtherTok{=} \DecValTok{1}
\NormalTok{  somme }\OtherTok{=} \DecValTok{0}
  \ControlFlowTok{if}\NormalTok{(k}\SpecialCharTok{\textgreater{}=}\DecValTok{0} \SpecialCharTok{\&}\NormalTok{  (p}\SpecialCharTok{\textgreater{}}\DecValTok{0} \SpecialCharTok{\&}\NormalTok{ p}\SpecialCharTok{\textless{}}\DecValTok{1}\NormalTok{) }\SpecialCharTok{\&}  \FunctionTok{is.numeric}\NormalTok{(k) }\SpecialCharTok{\&}\NormalTok{ k}\SpecialCharTok{\%\%}\DecValTok{1}\SpecialCharTok{==}\DecValTok{0}\NormalTok{)\{}
    \ControlFlowTok{while}\NormalTok{(i}\SpecialCharTok{\textless{}=}\NormalTok{k)\{}
\NormalTok{      probabilite }\OtherTok{=}\NormalTok{ p}\SpecialCharTok{*}\NormalTok{(}\DecValTok{1}\SpecialCharTok{{-}}\NormalTok{p)}\SpecialCharTok{**}\NormalTok{(i}\DecValTok{{-}1}\NormalTok{)}
\NormalTok{      somme }\OtherTok{=}\NormalTok{ somme }\SpecialCharTok{+}\NormalTok{ probabilite}
\NormalTok{      i }\OtherTok{=}\NormalTok{ i }\SpecialCharTok{+} \DecValTok{1}
\NormalTok{      \}}
\NormalTok{    \}}\ControlFlowTok{else}\NormalTok{\{}
    \FunctionTok{print}\NormalTok{(}\StringTok{"Entrer des valeurs correctes de p ou de k"}\NormalTok{)}
\NormalTok{    \}}
 \FunctionTok{print}\NormalTok{(somme)}
\NormalTok{\}}
\end{Highlighting}
\end{Shaded}
\end{frame}

\begin{frame}{Fonction de répartition de la loi géométrique}
\protect\hypertarget{fonction-de-ruxe9partition-de-la-loi-guxe9omuxe9trique-1}{}
\begin{center}\includegraphics{lois_discretes_files/figure-beamer/unnamed-chunk-25-1} \end{center}
\end{frame}

\begin{frame}[fragile]{Loi uniforme discrète}
\protect\hypertarget{loi-uniforme-discruxe8te}{}
Soit \(X\) une variable aléatoire discrète uniforme prenant \(n\)
valeurs \(k_1, k_2, \dots, k_n\). La fonction de probabilité de \(X\)
est donnée par :

\[P(X = k_i) = \frac{1}{n} \quad \text{pour } k_i \in \{k_1, k_2, \dots, k_n\}.\]

\begin{Shaded}
\begin{Highlighting}[]
\NormalTok{uniforme }\OtherTok{=} \ControlFlowTok{function}\NormalTok{(n)\{}
  \ControlFlowTok{if}\NormalTok{(n}\SpecialCharTok{\textgreater{}}\DecValTok{0} \SpecialCharTok{\&} \FunctionTok{is.numeric}\NormalTok{(n) }\SpecialCharTok{\&}\NormalTok{ n}\SpecialCharTok{\%\%}\DecValTok{1}\SpecialCharTok{==}\DecValTok{0}\NormalTok{)\{}
\NormalTok{    probabilite }\OtherTok{=} \DecValTok{1}\SpecialCharTok{/}\NormalTok{n}
    \FunctionTok{print}\NormalTok{(probabilite)}
\NormalTok{  \} }\ControlFlowTok{else}\NormalTok{\{}
    \FunctionTok{print}\NormalTok{(}\StringTok{"Entrer une valeur correcte de n"}\NormalTok{)}
\NormalTok{  \}}
\NormalTok{\}}
\end{Highlighting}
\end{Shaded}
\end{frame}

\begin{frame}[fragile]{Exemple}
\protect\hypertarget{exemple-4}{}
Considérons le jet d'un dé non biaisé. L'ensemble des \(n = 6\) valeurs
possible de \(X\) est \(A = {1, 2, 3, 4, 5, 6}\). A chaque fois que le
dé est jeté, la probabilité d'un résultat donné vaut \(1/6\).

\begin{Shaded}
\begin{Highlighting}[]
\NormalTok{n}\OtherTok{=}\DecValTok{6}
\FunctionTok{uniforme}\NormalTok{(n)}
\end{Highlighting}
\end{Shaded}

\begin{verbatim}
[1] 0.1666667
\end{verbatim}

Les caractéristiques de la loi uniforme discrète :
\(E(X) = \frac{n + 1}{2}\). La variance :
\(Var(X) = \frac{n^2 - 1}{12}\)
\end{frame}

\begin{frame}{Fonction de Répartition de la loi uniforme discrète}
\protect\hypertarget{fonction-de-ruxe9partition-de-la-loi-uniforme-discruxe8te}{}
La fonction de répartition \(F(k)\) pour \(X\) est définie comme suit :
\[F(k) = 
\begin{cases} 
0 & \text{pour } k < 1, \\
\frac{\lfloor k \rfloor}{n} & \text{pour } 1 \leq k < n, \\
1 & \text{pour } k \geq n.
\end{cases}\]

Exemple :

Considérons le jet d'un dé non biaisé. L'ensemble des \(n = 6\) valeurs
possible de \(X\) est \(A = {1, 2, 3, 4, 5, 6}\). A chaque fois que le
dé est jeté, la probabilité d'un résultat donné vaut \(1/6\). calculons
\(P(X\le 4)\). Cela revient à calculer \(F(4)= 4/6 = 2/3 = 0.66667\).
\end{frame}

\begin{frame}[fragile]{Fonction de Répartition de la loi uniforme
discrète}
\protect\hypertarget{fonction-de-ruxe9partition-de-la-loi-uniforme-discruxe8te-1}{}
\begin{Shaded}
\begin{Highlighting}[]
\NormalTok{repartition.unif.d }\OtherTok{=} \ControlFlowTok{function}\NormalTok{(n, k)\{}
  \ControlFlowTok{if}\NormalTok{((k}\SpecialCharTok{\textgreater{}=}\DecValTok{1} \SpecialCharTok{\&}\NormalTok{ k}\SpecialCharTok{\textless{}}\NormalTok{n) }\SpecialCharTok{\&}\NormalTok{ n}\SpecialCharTok{\textgreater{}}\DecValTok{0} \SpecialCharTok{\&} \FunctionTok{is.numeric}\NormalTok{(n) }\SpecialCharTok{\&}\NormalTok{ n}\SpecialCharTok{\%\%}\DecValTok{1}\SpecialCharTok{==}\DecValTok{0}\NormalTok{)\{}
\NormalTok{    probabilite }\OtherTok{=} \FunctionTok{floor}\NormalTok{(k)}\SpecialCharTok{/}\NormalTok{n}
    \FunctionTok{print}\NormalTok{(probabilite)}
\NormalTok{  \} }\ControlFlowTok{else}\NormalTok{\{}
    \ControlFlowTok{if}\NormalTok{(k}\SpecialCharTok{\textless{}}\DecValTok{1}\NormalTok{)\{}
      \FunctionTok{print}\NormalTok{(}\DecValTok{0}\NormalTok{)}
\NormalTok{    \} }\ControlFlowTok{else}\NormalTok{\{}
      \ControlFlowTok{if}\NormalTok{(k}\SpecialCharTok{\textgreater{}=}\NormalTok{n)\{}
        \FunctionTok{print}\NormalTok{(}\DecValTok{1}\NormalTok{)}
\NormalTok{      \}}\ControlFlowTok{else}\NormalTok{\{}
        \FunctionTok{print}\NormalTok{(}\StringTok{"Entrer une valeur correcte de n ou de k"}\NormalTok{)}
\NormalTok{      \}}
\NormalTok{    \}}
\NormalTok{  \}}
\NormalTok{\}}
\end{Highlighting}
\end{Shaded}
\end{frame}

\begin{frame}[fragile]{Application à l'exemple précédent}
\protect\hypertarget{application-uxe0-lexemple-pruxe9cuxe9dent}{}
\begin{Shaded}
\begin{Highlighting}[]
\FunctionTok{repartition.unif.d}\NormalTok{(}\DecValTok{6}\NormalTok{, }\DecValTok{4}\NormalTok{)}
\end{Highlighting}
\end{Shaded}

\begin{verbatim}
[1] 0.6666667
\end{verbatim}

Il y a donc 2 chances sur 3 que les valeurs des faces du dé soient
toutes inférieures ou égales à 4 (soit 67\%).

Enfin, tu peux tout simplement t'inspérer de tout cet arsenal pour
programmer les autres lois.
\end{frame}

\begin{frame}{Fonction de répartition de la loi uniforme}
\protect\hypertarget{fonction-de-ruxe9partition-de-la-loi-uniforme}{}
\begin{center}\includegraphics{lois_discretes_files/figure-beamer/unnamed-chunk-31-1} \end{center}
\end{frame}

\begin{frame}{Résumé}
\protect\hypertarget{ruxe9sumuxe9}{}
\begin{center}\includegraphics{lois_discretes_files/figure-beamer/unnamed-chunk-32-1} \end{center}
\end{frame}

\begin{frame}{Fin}
\protect\hypertarget{fin}{}
\begin{alertblock}{julienbidias246@gmail.com}

Bonne lecture et surtout n'hésite pas à écrire au mail ci-dessus pour faire des suggestions et avoir aussi le code complet avec tous les commentaires à l'appui.

\vspace{3mm}
A suivre...

\end{alertblock}
\end{frame}

\end{document}
